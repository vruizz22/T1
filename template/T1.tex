\documentclass[addpoints,10pt]{exam}

%Paquetes utilizados en esta tarea
\usepackage{fullpage}
\usepackage[utf8]{inputenc}
\usepackage[spanish]{babel}
\usepackage{epsfig}
\usepackage{amsmath}
\usepackage{amssymb}
\usepackage{epstopdf}
\usepackage[hidelinks]{hyperref}
\usepackage{algorithmic}
\usepackage[nothing]{algorithm}
\usepackage{color}
\usepackage{graphicx}
\usepackage{ragged2e}


% Colores de fondo de las soluciones
\shadedsolutions
\definecolor{SolutionColor}{rgb}{0.8,0.9,1}

% Colores de los boxes
% \colorsolutionboxes
% \definecolor{SolutionBoxColor}{gray}{1}


% Formato de las preguntas
\pointname{ Puntos}
\bracketedpoints
\pointsinleftmargin    

\qformat{\bf Pregunta \thequestion \hfill [\thepoints]\vspace{0.4cm}}


\printanswers
% \noprintanswers


%Definiciones de comandos, para reutilizar secuencias frecuentes o ahorrar
%código
\newcommand{\RR}{\mathbb{R}}
\newcommand{\ZZ}{\mathbb{Z}}
\newcommand{\mycomment}[1]{}
\newcommand{\lb}{\\~\\}
\newcommand{\la}{\leftarrow}
\newcommand{\R}{\mathbb{R}}
\newcommand{\IN}[2]{\in \{#1, \dots, #2\}}
\newcommand{\GP}[1]{{\color{cyan}[{\bf GP:\;}{#1}\ ]}}

\newcommand{\twopartdef}[4]
{
$
	\left\{
		\begin{array}{ll}
			#1 &  \text{#2} \\
			#3 &  \text{#4}
		\end{array}
	\right.
$
}

\newcommand{\threepartdef}[6]
{
	\left\{
		\begin{array}{ll}
			#1 &  \text{si }#2 \\
			#3 &  \text{si }#4 \\
			#5 &  \text{si }#6
		\end{array}
	\right.
}

\makeatletter

\makeatother

\begin{document}

\begin{tabular}{ccl}
\begin{tabular}{c}
\includegraphics[width=2.5cm]{logo/logo.pdf}
\end{tabular}
&\ \ \ & 
\begin{tabular}{l}
PONTIFICIA UNIVERSIDAD CATÓLICA DE CHILE\\
ESCUELA DE INGENIERÍA\\
DEPARTAMENTO DE INGENIERÍA INDUSTRIAL Y DE SISTEMAS\\
\end{tabular}
\end{tabular}

\begin{center}
\bf ICS1113 - Optimización\\
\bf 1er semestre del 2024\lb

\vspace{0.5cm}

\bf {\Huge Tarea 1 ICS1113}
\end{center}

\setcounter{secnumdepth}{0} % desactiva la numeración de las secciones

	
	\begin{flushleft}
		
		\section{Problema 1}
		\subsection{Conjuntos}
		\begin{itemize}
			\item $n \in \{1, \ldots, N\}$, donde $n$ pertenece al conjunto de puntos de demanda.
			\item $t \in \{1, \ldots, T\}$, donde $t$ pertenece al conjunto de d\'ias para cumplir la demanda.
			\item $b \in \{1, \ldots, \Theta\}$, donde $b$ pertenece al conjunto de parques de estacionamiento.
			\item $\gamma \in \{1, \ldots, \Gamma\}$, donde $\gamma$ pertenece al conjunto de ubicaciones de los centros de carga.
			\item $W_{m^3}$ de agua para transportar en cada viaje.
		\end{itemize}

		\subsection{Parámetros}
		\begin{itemize}
			\item $C_{bn}$, costo de viajar desde $b$ hasta $n$.
			\item $CL_b$, tarifa de pasar la noche en el estacionamiento $b$.
			\item $D_{nt}$, el tiempo de viaje estimado entre las ubicaciones $i$ y $j$.
			\item $(b, n) = a \in A$, conjunto de arcos $(b, n)$ que representan las rutas desde el parque $b$ hasta cualquier punto de demanda $n \in N$.
			\item $(n, b) = v \in V$, conjunto de arcos $(n, b)$ que representan las rutas desde el punto de demanda $n$ hasta cualquier parque de estacionamiento $b \in \Theta$.
		\end{itemize}

		\subsection{Variables de decisión} 
		\begin{itemize}
			\item \[
				\epsilon_{\gamma t\alpha} = 
					 \begin{cases}
					   1 &\quad\text{si el camión }\gamma\text{ en el día }t\text{ va a entregar agua en la ruta }a\\
					   0 & \quad\text{en cualquier otro caso.}
					 \end{cases}
				\]
			\item \[
				k_{\gamma b t} = 
					 \begin{cases}
					   1 &\quad\text{si el camión }\gamma\text{ se queda en la noche en el parque }b\text{ en el día }t\\
					   0 &\quad\text{en cualquier otro caso.}
					 \end{cases}
				\]
				
			\item \[
				\epsilon_{\gamma t v} = 
					 \begin{cases}
					   1 &\quad\text{si el camión }\gamma\text{ el día }t\text{ toma la ruta }v\\
					   0 &\quad\text{en cualquier otro caso.}
					 \end{cases}
				\]
			\item $q_{\gamma t}$, cantidad de agua transportada por el camión $\gamma$ en el día $t$.
			\item $z_{b}$, capacidad del parque $b \in \Theta$ de albergar camiones.
			\item $C_{a}$, costo de viaje desde $b$ a cualquier punto $n \in N$.
			\item $C_{v}$, costo de viaje desde $n$ a cualquier punto $b \in \Theta$. 
			\item $CL_{b}$, costo de pasar la noche en el parque $b \in \Theta$.
			\item  $d_{nt}$, demanda de agua en el punto $n \in N$ en el día $t \in T$.
		\end{itemize}

		\subsection{Función Objetivo}
		\begin{quote}
			\begin{center}
				Minimizar $\sum\limits_{t \in T}\sum\limits_{\gamma \in \Gamma}\sum\limits_{a \in A} C_{a} \cdot \epsilon_{\gamma t a} + \sum\limits_{t \in T}\sum\limits_{\gamma \in \Gamma}\sum\limits_{v \in V} C_{v} \cdot \epsilon_{\gamma t v} + \sum\limits_{t \in T}\sum\limits_{b \in \Theta} CL_{b} \cdot k_{\gamma b t}$	
			\end{center}
		\end{quote}

		\subsection{Restricciones}
		\begin{itemize}
			\item $ \sum\limits_{b \in \Theta} k_{\gamma b t} = 1 $ (Restricción de pasar la noche en un parque)
			\item $\sum\limits_{a \in A} \epsilon_{\gamma t\alpha} \leq 1 \quad \forall \gamma \in \Gamma, \; \forall t \in T$ (Restricción de cantidad de viajes por día)
			\item $\sum\limits_{v \in V} \epsilon_{\gamma t v} = 1 \quad \forall \gamma \in \Gamma, \; \forall t \in T$ (Restricción de volver al parque)
			\item $\sum\limits_{n \in N} d_{nt} = \sum\limits_{\gamma \in \Gamma} q_{\gamma t} \quad \forall t \in T$ (Restricción de demanda de agua)
			\item $q_{\gamma t} \leq W_{m^3} \quad \forall t \in T, \; \forall \gamma \in \Gamma$ (Restricción de capacidad de agua)
			\item $Z_{b} \geq \Gamma$ (Restricción de capacidad de parque)
		\end{itemize}

		\subsection{Naturaleza de las variables}
		\begin{itemize}
			\item $Z_{b} \geq 0 \quad \forall b \in \Theta$
			\item $q_{\gamma t} \geq 0 \quad \forall t \in T, \; \forall \gamma \in \Gamma$
			\item $d_{nt} \geq 0 \quad \forall t \in T, \; \forall n \in N$
			\item $C_{a}, \; C_{v}, \; CL_{b} \geq 0$
		\end{itemize}
		
		\section{Problema 2}
			

		\subsection{Conjuntos}
		\begin{itemize}
			\item $n \in \{1, \ldots, N\}$, número de planta generadora.
			\item $m \in \{1, \ldots, M\}$, número de nodos de consumo electrico.
			\item $h \in \{1, \ldots, H\}$, número de horas del día.
			\item $\Theta = \{(n,m) \in N \times M\}$, conjunto de arcos que representan la transmisión de energía entre las plantas generadoras y los nodos de consumo.
		\end{itemize}
		
		\subsection{Parámetros}
		\begin{itemize}
			\item $D_{mh}$, demanda de energía en el nodo $m$ en la hora $h$.
			\item $CN_{nh}$, capacidad de producción de energía de la planta $n$ en la hora $h$.
			\item $R_{nm}$, coeficiente de perdida de energía al transmitir energía desde la planta $n$ al nodo $m$.
			\item $G_{nm}$, factor para obtener la perdida no óhmica de energía al transmitir energía desde la planta $n$ al nodo $m$.
			\item $CL_{nm}$, capacidad máxima de energía para cada ruta $(n,m) \in \Theta$.
		\end{itemize}
		\subsection{Variables de decisión} 
		\begin{itemize}
			\item $E_{n,m},h$, energía enviada desde la planta $n$ al nodo $m$ en la hora $h$.
			\item $c_{n,h}$, costo de producción de energía en la planta $n$ desde la hora $h$ hasta el final del día.
			\item \[
				P_{n,o} = 
					 \begin{cases}
					   1 &\quad\text{si se prende la generadora }n\text{ en la hora }o\\
					   0 &\quad\text{en cualquier otro caso.}
					 \end{cases}
				\]
			\item $EF_(n,m),h$, energía que llega al nodo $m$ en la hora $h$ tras las perdidas en la ruta $(n,m)$.
			\item $CMeV_{L}$, costo medio variable de generar $L$ unidades de energía $(MWh)$.
		\end{itemize}
		\subsection{Función Objetivo}
		\begin{quote}
			\begin{center}
				Min $\sum\limits_{n \in N} c_{n,o} \cdot P_{n,o} + \sum\limits_{h \in H} (\sum\limits_{n \in N} ((a_{n,h} - b_{n,h} \cdot L) \cdot \sum\limits_{m \in M} E_{nm},h))$
			\end{center}
		\end{quote}
		\subsection{Restricciones}
		\begin{itemize}
			\item $\sum\limits_{n \in N} E_{nm},h = D_{mh} \quad \forall m \in M, \; \forall h \in H$ (Restricción de demanda de energía)
			\item $\sum\limits_{m \in M} E_{nm},h \leq CN_{nh} \quad \forall n \in N, \; \forall h \in H$ (Restricción de capacidad de producción)
			\item $E_{nm},h \leq CL_{nm} \quad \forall n \in N, \; \forall m \in M, \; \forall h \in H$ (Restricción de capacidad de transmisión)
		\end{itemize}
		\subsection{Naturaleza de las variables}
		\begin{itemize}
			\item $EF_{n,m,h}, \; E_{n,m,h} \geq 0 \quad \forall n \in N, \; \forall m \in M, \; \forall h \in H$
			\item  $D_{m,h}, \; CN_{n,h}, \; R_{n,m}, \; G_{n,m}, \; CL_{n,m} \geq 0$
			\item  $CMeV_{L} \geq 0$
			\item 
		\end{itemize}
	
		
	\end{flushleft}
	
\end{document}