\documentclass[addpoints,10pt]{exam}

%Paquetes utilizados en esta tarea
\usepackage{fullpage}
\usepackage[utf8]{inputenc}
\usepackage[spanish]{babel}
\usepackage{epsfig}
\usepackage{amsmath}
\usepackage{amssymb}
\usepackage{epstopdf}
\usepackage[hidelinks]{hyperref}
\usepackage{algorithmic}
\usepackage[nothing]{algorithm}
\usepackage{color}
\usepackage{graphicx}
\usepackage{ragged2e}


% Colores de fondo de las soluciones
\shadedsolutions
\definecolor{SolutionColor}{rgb}{0.8,0.9,1}

% Colores de los boxes
% \colorsolutionboxes
% \definecolor{SolutionBoxColor}{gray}{1}


% Formato de las preguntas
\pointname{ Puntos}
\bracketedpoints
\pointsinleftmargin    

\qformat{\bf Pregunta \thequestion \hfill [\thepoints]\vspace{0.4cm}}


\printanswers
% \noprintanswers


%Definiciones de comandos, para reutilizar secuencias frecuentes o ahorrar
%código
\newcommand{\RR}{\mathbb{R}}
\newcommand{\ZZ}{\mathbb{Z}}
\newcommand{\mycomment}[1]{}
\newcommand{\lb}{\\~\\}
\newcommand{\la}{\leftarrow}
\newcommand{\R}{\mathbb{R}}
\newcommand{\IN}[2]{\in \{#1, \dots, #2\}}
\newcommand{\GP}[1]{{\color{cyan}[{\bf GP:\;}{#1}\ ]}}

\newcommand{\twopartdef}[4]
{
$
	\left\{
		\begin{array}{ll}
			#1 &  \text{#2} \\
			#3 &  \text{#4}
		\end{array}
	\right.
$
}

\newcommand{\threepartdef}[6]
{
	\left\{
		\begin{array}{ll}
			#1 &  \text{si }#2 \\
			#3 &  \text{si }#4 \\
			#5 &  \text{si }#6
		\end{array}
	\right.
}

\makeatletter

\makeatother

\begin{document}

\begin{tabular}{ccl}
\begin{tabular}{c}
\includegraphics[width=2.5cm]{logo/logo.pdf}
\end{tabular}
&\ \ \ & 
\begin{tabular}{l}
PONTIFICIA UNIVERSIDAD CATÓLICA DE CHILE\\
ESCUELA DE INGENIERÍA\\
DEPARTAMENTO DE INGENIERÍA INDUSTRIAL Y DE SISTEMAS\\
\end{tabular}
\end{tabular}

\begin{center}
\bf ICS1113 - Optimización\\
\bf 1er semestre del 2024\

\vspace{0.5cm}

\bf {\Huge Tarea 1 ICS1113}
\end{center}

\setcounter{secnumdepth}{0} % desactiva la numeración de las secciones
\begin{flushleft}
\begin{justify}
\section{Sección}
\subsection{Subsección}
Texto de ejemplo. Aqui escribe tu tarea.
\end{justify}
\end{flushleft}
\end{document}